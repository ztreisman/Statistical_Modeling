\documentclass[12pt,fullpage]{amsart}
\usepackage{enumerate,graphicx,fullpage, multicol}

\input{/home/treisman/tex/macros}


\pagestyle{empty}
\setlength{\parindent}{0pt}
\newcommand{\ds}{\displaystyle}
\begin{document}

\textbf{Math 213 \quad\ddag\quad Practice Test \#2 \quad\ddag\quad Fall 2021}

\hrulefill
\medskip
\begin{enumerate}

\item Which of the following statements about probability distributions is/are \underline{true}?
  \begin{enumerate}
  \item The area under the entire distribution curve is 1.
  \item The distribution is never negative.
  \item All distributions are symmetric.
  \item 95\% of the area under the distribution curve is within 2 standard deviations of the mean.
  \end{enumerate}

\vfill

\item The length of the thorax of a population of fruit flies is
  normally distributed with mean 0.8 mm and standard deviation 0.078
  mm.
  \begin{enumerate}
  \item What proportion of the fruit flies have thorax length less
    than 0.72 mm?
  \item What proportion of the fruit flies have thorax length greater
    than 0.82 mm?
  \item What proportion of the fruit flies have thorax length between
    0.7 and 0.9 mm?
  \item We wish to select the fruit flies with the highest 20\% of
    thorax length.  What is the shortest thorax length we should
    consider?
  \end{enumerate}

  \vfill
 
  
\item Which of the following statements about z-scores is/are \underline{true}?
\begin{enumerate}
\item	larger z-scores are always better
\item	the z-score for an observation that is equal to the mean is 1
\item	if a z-score is 2 that means that the observation is two times the mean
\item	if a z-score is negative that means that the observation is less than the mean
\item   none of the above are true
\end{enumerate}


      \vfill
      
\item
The distribution of rhesus monkey tail lengths is bell-shaped, unimodal, and approximately symmetric.  The average tail length is 6.8 cm and the standard deviation is 0.44 cm.  Roscoe has a tail that is 10.2 cm long.  What conclusion can we make based on the information given?
\begin{enumerate}
    \item
    We can apply the empirical rule to conclude that Roscoe is a potential outlier because he falls more than three standard deviations away from the mean.
    \item
	We can apply the empirical rule to conclude that Roscoe is not a potential outlier because he falls within three standard deviations away from the mean.
    \item
	We cannot apply the empirical rule because the distribution does not fit the criteria for the empirical rule.
    \item
      There is not enough information given to make any conclusions about potential outliers.

\end{enumerate} 

      \vfill
      
\item
Based on a random sample of 120 rhesus monkeys, a 95\% confidence interval for the proportion of rhesus monkeys that live in a captive breeding facility and were assigned to research studies is (0.67, 0.83).  Which of the following is \underline{true}?
\begin{enumerate}
\item	95 of the sampled monkeys were assigned to research studies
\item	the margin of error for the confidence interval is 0.16
\item	a larger sample size would yield a wider confidence interval
\item	if we used a different confidence level, the interval would not be symmetric about the sample proportion
\item	none of the above are true
\end{enumerate}

\vfill

\item
  Approximately 19\% of physics majors in the US are women. A random sample of 50 physics majors from all Colorado universities with majors in physics includes 23 females.
  \begin{enumerate}
  \item What is your point estimate for the proportion of Colorado physics majors who are female?
    \item Justify the use of a normal model to do inference on this proportion. 
  \item Using a normal model for the proportion, what is the standard error in your estimate?
    \item Give a 95\% confidence interval for the proportion of Colorado physics majors who are female.
    \item If you would like your margin of error to be at most $\pm 5\%$, how many physics majors would you have to include in your sample?
  \end{enumerate}

  \vfill
  
\item
The World Bank reports that 1.7\% of the US population lives on less than \$2 per day.  A policy maker claims that this number is misleading because of variation from state to state and rural to urban. To investigate this, she takes a random sample of 100 households in Atlanta to compare with the national average and finds that 2.1\% of the Atlanta population live on less than \$2/day. Select the null and alternative hypothesis to test whether Atlanta differs significantly from the national percentage.
\begin{enumerate}
\item $H_0$: $p= 2.1$,   $H_a$: $p \neq 2.1$
\item $H_0$: $\mu=0.021$, $H_a$: $\mu \neq 0.021$
\item $H_0$: $p=1.7$,	  $H_a$: $p \neq 1.7$
\item $H_0$: $p= 0.017$, $H_a$: $p \neq 0.017$
\item $H_0$: $\mu = 2$, $H_a$: $\mu \neq 2$
\end{enumerate}

\vfill

\item
Complete the following sentence: When conducting a hypothesis test, we \underline{\hspace{1in}} and then evaluate the test results to determine if there is enough evidence to \underline{\hspace{1in}}.
\begin{enumerate}
\item	Assume that the null hypothesis is false; accept the null hypothesis
\item	Assume that the null hypothesis is true; reject the null hypothesis
\item	Assume that the alternative hypothesis is true; reject the null hypothesis
\item	Assume the alternative hypothesis is false; reject the alternative hypothesis
\end{enumerate}

      \vfill
      
\item Approximately 8\% of Colorado residents have been infected with COVID-19. Which of the following are true?
  \begin{enumerate}
  \item If we take samples of size 20, the sampling distribution for the proportion of Coloradans who have been infected with COVID-19 will be right skewed.
  \item If we take samples of size 200, the sampling distribution for the proportion of Coloradans who have been infected with COVID-19 will be right skewed.
  \item A sample of 200 Coloradans of whom 50 have been infected with COVID-19 would be considered unusual.
  \item A sample of 200 Coloradans of whom 20 have been infected with COVID-19 would be considered unusual.
  \end{enumerate}
    

      
\item
A psychologist wants to determine if socioeconomic status is related to game playing preferences.  Sixty children, in total, were identified from families of low, middle, and high socioeconomic status (20 each), and then the children were asked to select one of Monopoly, Battleship, or Connect Four. The psychologist computed the test statistic $\chi^2=5.2$. The proportion of a theoretical $\chi^2$ distribution with 4 degrees of freedom that is greater than $5.2$ is approximately $0.2674$. What can we say about the $p$-value, $H_0$, and the conclusion at the $\alpha=0.05$ level of significance?

\begin{enumerate}
    \item
    $0.05 < p\mbox{-value} < 0.1$; reject $H_0$; there is evidence of an association between socioeconomic status and game preference
    \item
    $p\mbox{-value} > 0.3$; fail to reject $H_0$; no evidence of an association between socioeconomic status and game preference
    \item
    $0.2 < p\mbox{-value} < 0.3$; fail to reject $H_0$; no evidence of an association between socioeconomic status and game preference
    \item
    $0.2 < p\mbox{-value} < 0.3$; fail to reject $H_0$; there is evidence of an association between socioeconomic status and game preference
    \item
    $0.05 < p\mbox{-value} < 0.1$; fail to reject $H_0$; no evidence of an association between socioeconomic status and game preference
\end{enumerate}

\vfill


\item A coin is flipped 1000 times. It comes up heads 532 times. Is this a fair coin?
  \begin{enumerate}
  \item Give appropriate null and alternative hypotheses.
  \item Give the test statistic and $p$-value for the test.
\item Give a 95\% confidence interval for the probability that the coin comes up heads.
  \item Clearly interpret your results in a sentence.
  \end{enumerate}

  \vfill
  
\item
The table below describes residents of an Atlanta neighborhood based on their car ownership and public transportation usage. 
\begin{center}
\begin{tabular}{|l|cc|c|}
    \hline
                    & Owns car & Does not own car & Total \\
    \hline
    Uses public transport             & 34 & 94 & 128\\
    Does not use public transport     & 126 & 17 & 143\\
    \hline
    Total                             & 160 & 111 & 271\\
    \hline
\end{tabular}
\end{center}
\begin{enumerate}
\item  If there is no association between car ownership and public transportation usage, how many individuals would we expect to \emph{not} own a car and \emph{not} use public transport?
\item Perform a hypothesis test to analyze if car ownership and public transportation usage are independent.
\end{enumerate}

\vfill


\item An ecologist hypothesizes that a lake's fish population is stable when the ratios of two types of fish are 3:2. The ecologist samples the fish in the lake collects the following data.

\begin{center}
\begin{tabular}{ll}
\textbf{fish type} & \textbf{count}\\
\hline
fish A & 58\\
fish B & 22\\
\hline
\textbf{total} & 80
\end{tabular}
\end{center}

Do a hypothesis test to evaluate this model.
\begin{enumerate}
\item State your null hypotheses in words. 
  \item What test statistic could you calculate from the sample to assess the validity of your null hypothesis?
  \item How many of fish A do we expect to find out of 80 total fish if the 3:2 model is correct?
    \item State your null hypothesis as a mathematical expression. ($H_0: \ldots$)
    \item What is the expected sampling distribution of your test statistic if your null hypothesis is true? 
  \item What are the observed value of the test statistic and the $p$-value from your hypothesis test?
\item What is your conclusion based on your test?
  \end{enumerate}

\vfill

\item An ecologist wants to know if the distributions of two types of fish are the same in two lakes. She collects the following data.

\begin{center}
\begin{tabular}{llll}
\textbf{fish type} & \textbf{Blue Lake} & \textbf{Green Lake} & \textbf{totals}\\
\hline
fish A & 65 & 40 & 105\\
fish B & 41 & 34 & 75\\
\hline
\textbf{totals} & 106 & 74 & 180
\end{tabular}
\end{center}

Do a hypothesis test to answer the ecologist's question.
\begin{enumerate}
\item State your null hypotheses in words. 
  \item What test statistic could you calculate from the sample to assess the validity of your null hypothesis?
  \item How many of fish A do we expect to see in Green Lake if the distributions are the same in both lakes?
    \item State your null hypothesis as a mathematical expression. ($H_0: \ldots$)
    \item What is the expected sampling distribution of your test statistic if your null hypothesis is true? 
  \item What are the observed value of the test statistic and the $p$-value from your hypothesis test?
\item What is your conclusion based on your test?
\end{enumerate}

      \vfill
      
\item In many sports, teams compete in seven game series, where games are played until one team has won four total games. A seven game series takes at most seven games. A theoretical model used to predict the length of a seven game series says that evenly matched teams will conclude the series in 4 games 12.5\% of the time, in 5 games 25\% of the time, in 6 games 31.25\% of the time and in 7 games 31.25\% of the time.

  During the years 1990-2019, 87 semifinal and final (World Series) baseball series were played in Major League Baseball. Of those series, 15 ended in 4 games, 21 in 5, 29 in 6 games, and 22 took all 7 games to conclude the series.

  \begin{enumerate}
  \item If the theoretical model is correct, how many of the 87 series would we expect to have ended in 5 games?
  \item A chi-squared test comparing these counts to the counts expected from the theoretical model gives a p value of 0.44. Which of the following is/are \underline{true}?
    \begin{enumerate}
    \item The theoretical model does not apply to Major League Baseball.
    \item The difference in the number of series that took five games to complete and the number predicted by the theoretical model is statistically significant.
    \item The distribution of series lengths from Major League Baseball is not inconsistent with the theoretical model.
    \item The theoretical model is useful for predicting series lengths.
    \item None of the above are true.
    \end{enumerate}
  \end{enumerate}

\vfill



\end{enumerate}

\end{document}


%%%%%%%%%%%%%%%%%%%%%%%%%%%%%%%%%%%%%%%%%%%%%%%%%%%%%%%%%%%%%%%
%%%%%%%%%%%%%%%%%%%%%%%%%%%%%%%%%%%%%%%%%%%%%%%%%%%%%%%%%%%%%%%%

\item
Suppose a t-test yields a $p$-value of 0.03 when testing hypotheses $H_0$: $\mu=0$ vs $H_a$: $\mu \neq 0$. Which of the following would be plausible 95\% confidence interval for $\mu$?
\begin{enumerate}
\item	(1.3, 3.4)
\item	(-3.4, -1.3)
\item	(-1.3, 3.4)
\item	Not enough information to determine
\end{enumerate}

\vfill

\item For the following R output, which of the following is \underline{true}?
\begin{verbatim}
t = 0.2024, df = 14, p-value = 0.8425
alternative hypothesis: true mean is not equal to 20
95 percent confidence interval:
 18.91483 21.31132
sample estimates:
mean of x
 20.11307
\end{verbatim}

\begin{enumerate}
\item	This is a one-sided test.
\item	At $\alpha=0.05$, we reject the null hypothesis.
\item	There is a 0.16 probability that the null hypothesis is false.
\item	This analysis had a sample size of $n=14$.
\item   None of the above are true.
\end{enumerate}

\newpage

\item Researchers measured the metabolic rate (in kCal/day) for 12 participants in a study on diet and exercise.  The measurements recorded were as follows: 995 1425 1396 1418 1502 1256 1189  913 1124 1052 1347 1204.

  The researchers would like to know if the metabolic rates for this group differ from the presumed baseline metabolic rate of 1500 kCal/day. 
  
  \begin{enumerate}
  \item Give appropriate null and alternative hypotheses.
  \item Are the conditions for inference using a t distribution satisfied?
  \item What is the t statistic?
  \item What are the degrees of freedom?
  \item What is the $p$-value for the hypotheis test?
  \item At the $\alpha=0.05$ significance level, is this a statistically significant result?
  \item Give a 95\% confidence interval for the mean metabolic rate in this group.
  \item Clearly interpret your results in a sentence.
  \end{enumerate}

\vfill
  
  \item Veterinarians at a nonhuman primate research center are interested in estimating the true average birth weight of rhesus monkeys born in captivity.  Below are the summary statistics of the data and output from the analysis testing if the true average birth weight of the monkeys is 0.4kg.  What is the correct calculation to estimate the true average birth weight of rhesus monkeys with a 95\% confidence interval?
\begin{verbatim}
 min   Q1 median  Q3  max mean   sd  n
0.27 0.37   0.39 0.5 0.68 0.44 0.12 10

t = 1.0853, df = 9, p-value = 0.306
alternative hypothesis: true mean is not equal to 0.4
\end{verbatim}
\begin{enumerate}
\item	$0.44 \pm 1.0853 \times 0.12 / \sqrt{9} $
\item	$0.44 \pm 1.0853 \times 0.12  $
\item	$0.44 \pm 2.26 \times 0.12 / \sqrt{10} $
\item	$0.39 \pm 1.96 \times 0.12 $
\item  $0.39 \pm 2.26 \times 0.12 / \sqrt{9} $
\end{enumerate}

\end{enumerate}

\end{document}
