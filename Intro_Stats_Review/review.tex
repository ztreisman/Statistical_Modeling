% Options for packages loaded elsewhere
\PassOptionsToPackage{unicode}{hyperref}
\PassOptionsToPackage{hyphens}{url}
\PassOptionsToPackage{dvipsnames,svgnames,x11names}{xcolor}
\documentclass[
  ignorenonframetext,
]{beamer}
\newif\ifbibliography
\usepackage{pgfpages}
\setbeamertemplate{caption}[numbered]
\setbeamertemplate{caption label separator}{: }
\setbeamercolor{caption name}{fg=normal text.fg}
\beamertemplatenavigationsymbolsempty
% remove section numbering
\setbeamertemplate{part page}{
  \centering
  \begin{beamercolorbox}[sep=16pt,center]{part title}
    \usebeamerfont{part title}\insertpart\par
  \end{beamercolorbox}
}
\setbeamertemplate{section page}{
  \centering
  \begin{beamercolorbox}[sep=12pt,center]{section title}
    \usebeamerfont{section title}\insertsection\par
  \end{beamercolorbox}
}
\setbeamertemplate{subsection page}{
  \centering
  \begin{beamercolorbox}[sep=8pt,center]{subsection title}
    \usebeamerfont{subsection title}\insertsubsection\par
  \end{beamercolorbox}
}
% Prevent slide breaks in the middle of a paragraph
\widowpenalties 1 10000
\raggedbottom
\AtBeginPart{
  \frame{\partpage}
}
\AtBeginSection{
  \ifbibliography
  \else
    \frame{\sectionpage}
  \fi
}
\AtBeginSubsection{
  \frame{\subsectionpage}
}
\usepackage{iftex}
\ifPDFTeX
  \usepackage[T1]{fontenc}
  \usepackage[utf8]{inputenc}
  \usepackage{textcomp} % provide euro and other symbols
\else % if luatex or xetex
  \usepackage{unicode-math} % this also loads fontspec
  \defaultfontfeatures{Scale=MatchLowercase}
  \defaultfontfeatures[\rmfamily]{Ligatures=TeX,Scale=1}
\fi
\usepackage{lmodern}
\ifPDFTeX\else
  % xetex/luatex font selection
\fi
% Use upquote if available, for straight quotes in verbatim environments
\IfFileExists{upquote.sty}{\usepackage{upquote}}{}
\IfFileExists{microtype.sty}{% use microtype if available
  \usepackage[]{microtype}
  \UseMicrotypeSet[protrusion]{basicmath} % disable protrusion for tt fonts
}{}
\makeatletter
\@ifundefined{KOMAClassName}{% if non-KOMA class
  \IfFileExists{parskip.sty}{%
    \usepackage{parskip}
  }{% else
    \setlength{\parindent}{0pt}
    \setlength{\parskip}{6pt plus 2pt minus 1pt}}
}{% if KOMA class
  \KOMAoptions{parskip=half}}
\makeatother
\usepackage{color}
\usepackage{fancyvrb}
\newcommand{\VerbBar}{|}
\newcommand{\VERB}{\Verb[commandchars=\\\{\}]}
\DefineVerbatimEnvironment{Highlighting}{Verbatim}{commandchars=\\\{\}}
% Add ',fontsize=\small' for more characters per line
\usepackage{framed}
\definecolor{shadecolor}{RGB}{248,248,248}
\newenvironment{Shaded}{\begin{snugshade}}{\end{snugshade}}
\newcommand{\AlertTok}[1]{\textcolor[rgb]{0.94,0.16,0.16}{#1}}
\newcommand{\AnnotationTok}[1]{\textcolor[rgb]{0.56,0.35,0.01}{\textbf{\textit{#1}}}}
\newcommand{\AttributeTok}[1]{\textcolor[rgb]{0.13,0.29,0.53}{#1}}
\newcommand{\BaseNTok}[1]{\textcolor[rgb]{0.00,0.00,0.81}{#1}}
\newcommand{\BuiltInTok}[1]{#1}
\newcommand{\CharTok}[1]{\textcolor[rgb]{0.31,0.60,0.02}{#1}}
\newcommand{\CommentTok}[1]{\textcolor[rgb]{0.56,0.35,0.01}{\textit{#1}}}
\newcommand{\CommentVarTok}[1]{\textcolor[rgb]{0.56,0.35,0.01}{\textbf{\textit{#1}}}}
\newcommand{\ConstantTok}[1]{\textcolor[rgb]{0.56,0.35,0.01}{#1}}
\newcommand{\ControlFlowTok}[1]{\textcolor[rgb]{0.13,0.29,0.53}{\textbf{#1}}}
\newcommand{\DataTypeTok}[1]{\textcolor[rgb]{0.13,0.29,0.53}{#1}}
\newcommand{\DecValTok}[1]{\textcolor[rgb]{0.00,0.00,0.81}{#1}}
\newcommand{\DocumentationTok}[1]{\textcolor[rgb]{0.56,0.35,0.01}{\textbf{\textit{#1}}}}
\newcommand{\ErrorTok}[1]{\textcolor[rgb]{0.64,0.00,0.00}{\textbf{#1}}}
\newcommand{\ExtensionTok}[1]{#1}
\newcommand{\FloatTok}[1]{\textcolor[rgb]{0.00,0.00,0.81}{#1}}
\newcommand{\FunctionTok}[1]{\textcolor[rgb]{0.13,0.29,0.53}{\textbf{#1}}}
\newcommand{\ImportTok}[1]{#1}
\newcommand{\InformationTok}[1]{\textcolor[rgb]{0.56,0.35,0.01}{\textbf{\textit{#1}}}}
\newcommand{\KeywordTok}[1]{\textcolor[rgb]{0.13,0.29,0.53}{\textbf{#1}}}
\newcommand{\NormalTok}[1]{#1}
\newcommand{\OperatorTok}[1]{\textcolor[rgb]{0.81,0.36,0.00}{\textbf{#1}}}
\newcommand{\OtherTok}[1]{\textcolor[rgb]{0.56,0.35,0.01}{#1}}
\newcommand{\PreprocessorTok}[1]{\textcolor[rgb]{0.56,0.35,0.01}{\textit{#1}}}
\newcommand{\RegionMarkerTok}[1]{#1}
\newcommand{\SpecialCharTok}[1]{\textcolor[rgb]{0.81,0.36,0.00}{\textbf{#1}}}
\newcommand{\SpecialStringTok}[1]{\textcolor[rgb]{0.31,0.60,0.02}{#1}}
\newcommand{\StringTok}[1]{\textcolor[rgb]{0.31,0.60,0.02}{#1}}
\newcommand{\VariableTok}[1]{\textcolor[rgb]{0.00,0.00,0.00}{#1}}
\newcommand{\VerbatimStringTok}[1]{\textcolor[rgb]{0.31,0.60,0.02}{#1}}
\newcommand{\WarningTok}[1]{\textcolor[rgb]{0.56,0.35,0.01}{\textbf{\textit{#1}}}}
\usepackage{graphicx}
\makeatletter
\newsavebox\pandoc@box
\newcommand*\pandocbounded[1]{% scales image to fit in text height/width
  \sbox\pandoc@box{#1}%
  \Gscale@div\@tempa{\textheight}{\dimexpr\ht\pandoc@box+\dp\pandoc@box\relax}%
  \Gscale@div\@tempb{\linewidth}{\wd\pandoc@box}%
  \ifdim\@tempb\p@<\@tempa\p@\let\@tempa\@tempb\fi% select the smaller of both
  \ifdim\@tempa\p@<\p@\scalebox{\@tempa}{\usebox\pandoc@box}%
  \else\usebox{\pandoc@box}%
  \fi%
}
% Set default figure placement to htbp
\def\fps@figure{htbp}
\makeatother
\setlength{\emergencystretch}{3em} % prevent overfull lines
\providecommand{\tightlist}{%
  \setlength{\itemsep}{0pt}\setlength{\parskip}{0pt}}

\pgfdeclareimage[width=3.5cm]{mcslogo}{../images/western_logo_hor_MCS_3C_pos.pdf}
\pgfdeclareimage[width=1cm]{ccbysa}{../images/ccbysa88x31.png}
\titlegraphic{\href{http://creativecommons.org/licenses/by-sa/4.0/}{\pgfuseimage{ccbysa}}
\hfill
\href{https://western.edu/program/mathematics/}{\pgfuseimage{mcslogo}}}
%\usecolortheme{wcu}
%\institute{Western Colorado University}
%\setbeamertemplate{navigation symbols}{}
\usepackage{bookmark}
\IfFileExists{xurl.sty}{\usepackage{xurl}}{} % add URL line breaks if available
\urlstyle{same}
\hypersetup{
  pdftitle={Review of Introductory Statistics and Motivation for This Class},
  pdfauthor={Zack Treisman},
  colorlinks=true,
  linkcolor={Maroon},
  filecolor={Maroon},
  citecolor={blue},
  urlcolor={Blue},
  pdfcreator={LaTeX via pandoc}}

\title{Review of Introductory Statistics and Motivation for This Class}
\author{Zack Treisman}
\date{Spring 2026}

\begin{document}
\frame{\titlepage}

\begin{frame}{Philosophy}
\phantomsection\label{philosophy}
\begin{itemize}
\tightlist
\item
  multiple variables and high dimensional data,
\item
  situations where assuming normality is a bad idea,
\item
  classification problems,
\item
  and more.
\end{itemize}
\end{frame}

\begin{frame}{Winter Recreation: Modeling Multiple Variables}
\phantomsection\label{winter-recreation-modeling-multiple-variables}
Visitation rates as a function of

\begin{itemize}
\item new snow in the previous three days, 
\item modality,
\item trailhead location,
\item year. 
\end{itemize}

Motorized users head to Kebler, non-motorized users to Snodgrass. Some
shift in these patterns year to year is visible.

\pandocbounded{\includegraphics[keepaspectratio]{review_files/figure-beamer/unnamed-chunk-1-1.pdf}}
\end{frame}

\begin{frame}{SRA: Simulation and Nonparametric Models}
\phantomsection\label{sra-simulation-and-nonparametric-models}
Western's Strategic Resource Allocation process needed to avoid assuming
normal distributions. Instead, randomization determined how a program's
rankings compared with the rest of campus.

\pandocbounded{\includegraphics[keepaspectratio]{../images/MATH.pdf}}
\end{frame}

\begin{frame}{Forest Fire Recovery: Plant Community Classifications}
\phantomsection\label{forest-fire-recovery-plant-community-classifications}
Unsupervised learning discovered plant communities from transect data.
Gradient boosted regression trees detect these communities in satellite
data.

\pandocbounded{\includegraphics[keepaspectratio]{../images/brt_preds.png}}
\end{frame}

\begin{frame}[fragile]{Data frame \(\to\) Statistical analysis \(\to\)
Model}
\phantomsection\label{data-frame-to-statistical-analysis-to-model}
Data frames are arrangements of \textbf{observations} of
\textbf{variables}. An \textbf{observation} is a single unit. A
\textbf{variable} is a measurement made on that unit

\begin{itemize}
\tightlist
\item
  Record observations as \textbf{rows} and variables in
  \textbf{columns}.\\
\item
  Variables can be \textbf{categorical} or \textbf{numerical}.

  \begin{itemize}
  \tightlist
  \item
    Categorical variables can be \textbf{binary} or not,
    \textbf{ordered} or not.
  \item
    Numerical variables can be \textbf{discrete} or \textbf{continuous}.
  \end{itemize}
\item
  Dates, times and locations merit special consideration.
\item
  Vocabulary is not universal: Factor, case, treatment \ldots
\end{itemize}

\footnotesize

\begin{Shaded}
\begin{Highlighting}[]
\FunctionTok{head}\NormalTok{(Sitka) }\CommentTok{\# from package MASS}
\end{Highlighting}
\end{Shaded}

\begin{verbatim}
##   size Time tree treat
## 1 4.51  152    1 ozone
## 2 4.98  174    1 ozone
## 3 5.41  201    1 ozone
## 4 5.90  227    1 ozone
## 5 6.15  258    1 ozone
## 6 4.24  152    2 ozone
\end{verbatim}

\normalsize
\end{frame}

\begin{frame}{Distributions}
\phantomsection\label{distributions}
The \textbf{distribution} of a variable is a description of how often it
takes each possible value.

\begin{itemize}
\tightlist
\item
  An \textbf{observed distribution} is what we actually see.

  \begin{itemize}
  \tightlist
  \item
    ``the sample''
  \item
    column of a data frame
  \item
    summarized by computing means, standard deviations, medians, sample
    proportions, et cetera.
  \end{itemize}
\item
  A \textbf{theoretical distribution} is a mathematical guess.

  \begin{itemize}
  \tightlist
  \item
    ``assume \(X\) is normally distributed''
  \item
    ``assume the probability of success is 50\%''
  \item
    might come from a formula
  \item
    might come from randomization/ simulation
  \end{itemize}
\end{itemize}

Much of statistics is concerned with comparing observed distributions to
theoretical distributions.

We often discuss distributions of variables other than those explicitly
in our data, such as the mean of a variable in our data, a test
statistic like \(\chi^2\), or the residuals of a linear model.
\end{frame}

\begin{frame}[fragile]{Distributions of Categorical Variables}
\phantomsection\label{distributions-of-categorical-variables}
\begin{itemize}
\tightlist
\item
  The distribution of a categorical variable is a list of the percentage
  of observations in each category.
\end{itemize}

\begin{Shaded}
\begin{Highlighting}[]
\FunctionTok{table}\NormalTok{(Sitka}\SpecialCharTok{$}\NormalTok{treat)}\SpecialCharTok{/}\FunctionTok{length}\NormalTok{(Sitka}\SpecialCharTok{$}\NormalTok{treat)}
\end{Highlighting}
\end{Shaded}

\begin{verbatim}
## 
##   control     ozone 
## 0.3164557 0.6835443
\end{verbatim}

\begin{Shaded}
\begin{Highlighting}[]
\FunctionTok{barplot}\NormalTok{(}\FunctionTok{table}\NormalTok{(Sitka}\SpecialCharTok{$}\NormalTok{treat))}
\end{Highlighting}
\end{Shaded}

\pandocbounded{\includegraphics[keepaspectratio]{review_files/figure-beamer/unnamed-chunk-3-1.pdf}}
\end{frame}

\begin{frame}[fragile]{Distributions of Numeric Variables}
\phantomsection\label{distributions-of-numeric-variables}
\begin{itemize}
\tightlist
\item
  Picture the distribution of a numeric variable with a histogram,
  boxplot or density estimate.
\end{itemize}

\begin{Shaded}
\begin{Highlighting}[]
\FunctionTok{hist}\NormalTok{(Sitka}\SpecialCharTok{$}\NormalTok{size)}
\end{Highlighting}
\end{Shaded}

\includegraphics[height=0.5\textheight]{review_files/figure-beamer/unnamed-chunk-4-1}

\begin{itemize}
\tightlist
\item
  Shape: center, spread, skew, kurtosis
\end{itemize}
\end{frame}

\begin{frame}[fragile]{Summaries of Numeric Variables}
\phantomsection\label{summaries-of-numeric-variables}
\begin{Shaded}
\begin{Highlighting}[]
\FunctionTok{summary}\NormalTok{(Sitka}\SpecialCharTok{$}\NormalTok{size)}
\end{Highlighting}
\end{Shaded}

\begin{verbatim}
##    Min. 1st Qu.  Median    Mean 3rd Qu.    Max. 
##   2.230   4.345   4.900   4.841   5.400   6.630
\end{verbatim}

\begin{Shaded}
\begin{Highlighting}[]
\FunctionTok{quantile}\NormalTok{(Sitka}\SpecialCharTok{$}\NormalTok{size, }\FunctionTok{c}\NormalTok{(}\FloatTok{0.025}\NormalTok{,}\FloatTok{0.975}\NormalTok{))}
\end{Highlighting}
\end{Shaded}

\begin{verbatim}
##   2.5%  97.5% 
## 3.2370 6.2815
\end{verbatim}

\begin{Shaded}
\begin{Highlighting}[]
\FunctionTok{sd}\NormalTok{(Sitka}\SpecialCharTok{$}\NormalTok{size)}
\end{Highlighting}
\end{Shaded}

\begin{verbatim}
## [1] 0.7982084
\end{verbatim}

\begin{Shaded}
\begin{Highlighting}[]
\FunctionTok{var}\NormalTok{(Sitka}\SpecialCharTok{$}\NormalTok{size)}
\end{Highlighting}
\end{Shaded}

\begin{verbatim}
## [1] 0.6371367
\end{verbatim}

\begin{Shaded}
\begin{Highlighting}[]
\FunctionTok{IQR}\NormalTok{(Sitka}\SpecialCharTok{$}\NormalTok{size)}
\end{Highlighting}
\end{Shaded}

\begin{verbatim}
## [1] 1.055
\end{verbatim}
\end{frame}

\begin{frame}{A familiar theoretical distribution: the Normal
distribution}
\phantomsection\label{a-familiar-theoretical-distribution-the-normal-distribution}
\begin{itemize}
\tightlist
\item
  Sums of many independent effects are normally distributed.
\item
  Means are normally distributed.
\item
  Proportions of successes are eventually normally distributed.
\item
  Formula that you never use:
  \(\displaystyle N(\mu,\sigma^2)(x)=\frac{1}{\sigma\sqrt{2 \pi}}e^{-\frac{1}{2}\left(\frac{x-\mu}{\sigma}\right)^2}\)
\item
  The \textbf{standard normal distribution} has \(\mu=0\), \(\sigma=1\).
\item
  Observations on disparate scales can be standardized with z scores:
  \(\displaystyle z=\frac{x-\mu}{\sigma}\)
\end{itemize}

\includegraphics[width=\linewidth,height=0.35\textheight,keepaspectratio]{../images/6895997.pdf}
\end{frame}

\begin{frame}[fragile]{Randomization for a Theoretical Distribution}
\phantomsection\label{randomization-for-a-theoretical-distribution}
\scriptsize

\begin{Shaded}
\begin{Highlighting}[]
\NormalTok{n }\OtherTok{\textless{}{-}} \FunctionTok{nrow}\NormalTok{(Sitka); num\_ozone }\OtherTok{\textless{}{-}} \FunctionTok{sum}\NormalTok{(Sitka}\SpecialCharTok{$}\NormalTok{treat}\SpecialCharTok{==}\StringTok{"ozone"}\NormalTok{); num\_sim }\OtherTok{\textless{}{-}} \DecValTok{1000}
\FunctionTok{set.seed}\NormalTok{(}\DecValTok{17}\NormalTok{)                }\CommentTok{\# initialize the random number generator}
\NormalTok{diffs }\OtherTok{\textless{}{-}} \FunctionTok{numeric}\NormalTok{(num\_sim)   }\CommentTok{\# initialize a vector to hold the differences}
\ControlFlowTok{for}\NormalTok{(i }\ControlFlowTok{in} \DecValTok{1}\SpecialCharTok{:}\NormalTok{num\_sim)\{          }\CommentTok{\# loop: repeat the following num\_sim times}
\NormalTok{  ozone }\OtherTok{\textless{}{-}} \FunctionTok{sample}\NormalTok{(}\DecValTok{1}\SpecialCharTok{:}\NormalTok{n,num\_ozone)       }\CommentTok{\# 1. randomly select observations}
\NormalTok{  ozone\_mean }\OtherTok{\textless{}{-}} \FunctionTok{mean}\NormalTok{(Sitka}\SpecialCharTok{$}\NormalTok{size[ozone])}\CommentTok{\# 2. average selected observations}
\NormalTok{  control\_mean }\OtherTok{\textless{}{-}} \FunctionTok{mean}\NormalTok{(Sitka}\SpecialCharTok{$}\NormalTok{size[}\SpecialCharTok{{-}}\NormalTok{ozone])}\CommentTok{\# 3. average the others}
\NormalTok{  diffs[i] }\OtherTok{\textless{}{-}}\NormalTok{ ozone\_mean }\SpecialCharTok{{-}}\NormalTok{ control\_mean\}  }\CommentTok{\# 4. store difference in means}
\FunctionTok{hist}\NormalTok{(diffs)     }\CommentTok{\# plot the resulting distribution}
\end{Highlighting}
\end{Shaded}

\pandocbounded{\includegraphics[keepaspectratio]{review_files/figure-beamer/unnamed-chunk-10-1.pdf}}
\normalsize    
\end{frame}

\begin{frame}{Other Theoretical Distributions from Intro Stats}
\phantomsection\label{other-theoretical-distributions-from-intro-stats}
Your first statistics class introduced you to several useful
distributions:

\begin{itemize}
\tightlist
\item
  t - Like the normal distribution, but adjusted for describing means of
  small samples.
\item
  \(\chi^2\) - Sum of several squared standard normal distributions.
  Useful when discussing several proportions at once, such as when
  considering categorical variables with more than two possible values.
\item
  F - Similar to \(\chi^2\), used in ANOVA.
\end{itemize}

And maybe\ldots

\begin{itemize}
\tightlist
\item
  Binomial - How many successes in \(n\) trials?
\item
  Poisson - Count of discrete events in fixed time or space.
\item
  Perhaps others? We'll see lots more\ldots
\end{itemize}
\end{frame}

\begin{frame}{Sampling Distributions}
\phantomsection\label{sampling-distributions}
Given a data set (the sample) and a quantity that we can calculate from
the data (a sample or test statistic) we propose an expected
distribution for that calculated quantity (the sampling distribution).

\begin{itemize}
\tightlist
\item
  Sampling distributions are never observed, always theoretical.
\end{itemize}

Having a sampling distribution allows us to do inference.

\pandocbounded{\includegraphics[keepaspectratio]{review_files/figure-beamer/unnamed-chunk-11-1.pdf}}
\end{frame}

\begin{frame}[fragile]{Inference: Confidence Intervals}
\phantomsection\label{inference-confidence-intervals}
Often, we assume the shape of a sampling distribution, but not its
specific parameters. By using the data to estimate those parameters, we
get a guess at the sampling distribution that we can use to compute a
confidence interval.

\footnotesize

\begin{Shaded}
\begin{Highlighting}[]
\NormalTok{(}\AttributeTok{x\_bar =} \FunctionTok{mean}\NormalTok{(Sitka}\SpecialCharTok{$}\NormalTok{size))}
\end{Highlighting}
\end{Shaded}

\begin{verbatim}
## [1] 4.840785
\end{verbatim}

\begin{Shaded}
\begin{Highlighting}[]
\NormalTok{s }\OtherTok{=} \FunctionTok{sd}\NormalTok{(Sitka}\SpecialCharTok{$}\NormalTok{size); n }\OtherTok{=} \FunctionTok{nrow}\NormalTok{(Sitka); (}\AttributeTok{se =}\NormalTok{ s}\SpecialCharTok{/}\FunctionTok{sqrt}\NormalTok{(n))}
\end{Highlighting}
\end{Shaded}

\begin{verbatim}
## [1] 0.04016222
\end{verbatim}

\normalsize

\pandocbounded{\includegraphics[keepaspectratio]{review_files/figure-beamer/unnamed-chunk-13-1.pdf}}
\end{frame}

\begin{frame}[fragile]{Inference: Hypothesis Tests}
\phantomsection\label{inference-hypothesis-tests}
A null hypothesis determines a sampling distribution. Compare data to
that proposed distribution. If the data are sufficiently unlikely, we
reject the null hypothesis.

\scriptsize

\pandocbounded{\includegraphics[keepaspectratio]{review_files/figure-beamer/unnamed-chunk-14-1.pdf}}

\begin{verbatim}
## 
##  Welch Two Sample t-test
## 
## data:  size by treat
## t = 2.3163, df = 209.44, p-value = 0.02151
## alternative hypothesis: true difference in means between group control and group ozone is not equal to 0
## 95 percent confidence interval:
##  0.03144833 0.39086574
## sample estimates:
## mean in group control   mean in group ozone 
##              4.985120              4.773963
\end{verbatim}

\normalsize
\end{frame}

\begin{frame}[fragile]{The Key Question of Intro Stats:
\alert{Which test to use?!}}
\phantomsection\label{the-key-question-of-intro-stats}
One variable:

\begin{itemize}
\tightlist
\item
  Numeric: t-test for the mean (\texttt{t.test})
\item
  Binary categorical: z-test or binomial test for the proportion of
  success (\texttt{prop.test} or \texttt{binom.test})
\item
  Nonbinary categorical: \(\chi^2\) test for goodness of fit
  (\texttt{chisq.test})
\end{itemize}

Two variables:

\begin{itemize}
\tightlist
\item
  Both numeric: linear regression (\texttt{lm})
\item
  One numeric, one binary categorical: t-test for a difference of means
  (\texttt{t.test})
\item
  One numeric, one nonbinary categorical: analysis of variance
  (\texttt{lm} and \texttt{anova} or \texttt{aov})
\item
  Both binary categorical: z-test or binomial test for equality of
  proportions (\texttt{prop.test} or \texttt{binom.test})
\item
  Two categorical, at least one nonbinary: \(\chi^2\) test for
  independence (\texttt{chisq.test})
\end{itemize}
\end{frame}

\end{document}
